\documentclass[12pt,twoside]{article}
\usepackage{amsmath}
\usepackage[active]{srcltx}
\usepackage{amssymb}
\usepackage{amscd}
\usepackage{makeidx}
\usepackage{amsthm}
\usepackage{algorithm}
\usepackage{amssymb, amsmath}
\usepackage[utf8]{inputenc}
\usepackage{fancyhdr}
\usepackage{graphics}
\usepackage{amsmath, amssymb}
\usepackage{amsmath}
\usepackage{listings}
\usepackage{xcolor}
\usepackage{fancyhdr}
\usepackage[active]{srcltx}
\usepackage{amsmath}
\usepackage{amssymb}
\usepackage{amscd}
\usepackage{makeidx}
\usepackage{graphicx}
\usepackage{caption}
\usepackage{url}
\usepackage{enumitem}

\renewcommand{\baselinestretch}{1}
\setcounter{page}{1}
\setlength{\textheight}{21.6cm}
\setlength{\textwidth}{14cm}
\setlength{\oddsidemargin}{1cm}
\setlength{\evensidemargin}{1cm}
\pagestyle{myheadings}
\thispagestyle{empty}
\markboth{\small{TMP(Nombre de equipo)}}{\small{Formulación del Proyecto}}
\date{} 

\begin{document}
	
	\begin{center}
		
		% Contenido izquierdo - Imagen
		\begin{minipage}{0.17\textwidth}
			\flushleft
			\includegraphics[width=0.9\textwidth]{img/ipn_logo.png} % Ajusta esta línea
		\end{minipage}
		\begin{minipage}{.55\textwidth}
			\centering
			{\Large Instituto Politécnico Nacional}\\
			{\Large Escuela Superior de Cómputo}
		\end{minipage}
		\begin{minipage}{0.17\textwidth}
			\flushright
			\includegraphics[width=0.9\textwidth]{img/escom_logo} % Ajusta esta línea
		\end{minipage}			
	\end{center}
	
	
	\centerline{\bf Ingeniería en Inteligencia Artificial, Análisis y diseño de sistemas}
	
	\centerline{\bf Sem: 2024-2, 4BM2, TMP(Nombre de la actividad), Fecha: 27-06-24\\\\}
	
	\centerline{}
	
	%\centerline{}
	
	
	\begin{center}
		\Large{\textsc{Ecos del alma: Prototipo de juego rogue like con interacción jugador-NPC dinámica}} 
	\end{center}
	\centerline{}
	\centerline{\bf {\textit{Presentan}}}
	\centerline{}
	\centerline{\bf {Angeles López Erick Jesse\footnote{eangelesl1700@alumno.ipn.mx}}}
	\centerline{\bf {Corona Anchondo José Antonio\footnote{jcoronaa1900@alumno.ipn.mx}}}
	\centerline{\bf {Esquivel García Thania Paola\footnote{tesquivelg2000@alumno.ipn.mx}}}
	\centerline{\bf {	Espinosa Martínez José Carlos\footnote{jespinosam1900@alumno.ipn.mx}}}
	
	
	
	\newtheorem{Theorem}{\quad Theorem}[section]
	
	\newtheorem{Definition}[Theorem]{\quad Definition}
	
	\newtheorem{Corollary}[Theorem]{\quad Corollary}
	
	\newtheorem{Lemma}[Theorem]{\quad Lemma}
	
	\newtheorem{Example}[Theorem]{\quad Example}
	
	\bigskip
	
	\bigskip
	
	\textbf{Resumen:}  \\ 
	
	{\bf Palabras Clave:} \\
	
	\clearpage
	
	\tableofcontents
		
	\clearpage
		
	\section{Introducción}
	
	\clearpage
	\section{Estado del arte}%Videjuegos y aplicaciones que simulen este comportamiento conversacional
	Algunos juegos y aplicaciones similares son:
	\begin{enumerate}[noitemsep]
		\item Videojuego \textit{Hades} desarrollado por Supergiant Games \cite{game:hades}. 

		\item Videojuego \textit{Celeste} desarrollado por Maddy Makes Games \cite{game:celeste}

		\item Videojuego \textit{Hollow Knight} desarrollado por Team Cherry \cite{game:hollow}
		
		\item Videojuego \textit{Coffee Talk} desarrollado por Toge Productions \cite{game: coffee}
		
		\item Aplicación \textit{character.ai} desarrollada por Character Technologies, Inc. \cite{app: character}
	\end{enumerate}
	
	\begin{table}[H]
		\centering
		\begin{tabular}{|c|p{9cm}|}
			\hline
			\textbf{Producto} & \multicolumn{1}{c|}{\textbf{Características}} \\ \hline
			 Hades &  Juego \textit{rogue like}  con niveles generados proceduralmente conectando diferentes habitaciones.\\ \hline
			 
			 Celeste &  Juego de plataformas con jugabilidad invariable, conversación ocasional con NPC y mecanicas de movimiento avanzadas. \\ \hline
			 
			 Hollow knight &   Juego \textit{metroidvania} con poca interacción conversacional, mecanicas de movimiento y combate avanzada. \\ \hline
			 
			 Coffee talk &   Juego de genero novela visual y aventura conversacional, las respuestas (limitadas por el juego) definen las relaciones con los NPC. \\ \hline
			 
			 character.ai&   Aplicación online para diseñar o conversar con bots inteligentes y personificados unicamente mediante lineas de texto. \\ \hline
		\end{tabular}
		\caption{Resumen de juegos y aplicaciones similares}
		\label{table:aplicaciones}
	\end{table} 
	
	\clearpage
	\section{Justificación}
	
	\section{Objetivo general}
	
	\section{Objetivos particulares}
	
	\section{TMP(Capitulos de contexto)}
	
	\section{Análisis}
	
	\subsection{Metodología}

	\subsection{Tecnología}	% Herramientas, software, lenguajes de programación que seran utilizados para el desarrollo

	\subsubsection{Lenguaje de programación C\#}
	
	\subsubsection{Lenguaje de programación Prolog}
	
	\subsubsection{Motor de desarrollo Unity}
	
	\subsubsection{MongoDB}
	
	\subsubsection{Star UML}
	
	\subsection{Requerimientos}
	
	\subsubsection{Requerimientos funcionales}
	
	\subsubsection{Requerimientos no funcionales}
	
	\subsubsection{Requerimientos de sistema}

	\section{Diseño}
	
	\subsection{Diagrama de casos de uso}
	
	\subsubsection{Especificaciones}
	
	\subsection{Diagrama de clases}

	\subsection{Diagrama de secuencia}

	\subsection{Diagrama de estados}

	\subsection{Diagrama de actividades}

	\clearpage
	
	\section{Referencias Bibliográficas}
	
	\begin{thebibliography}{10}
	
	%Estado del arte
	\bibitem{game:hades}
	Supergiant Games, \textit{Hades}. Supergiant Games 2019.
	
	\bibitem{game:celeste}
	Maddy Makes Games , \textit{Celeste}. Maddy Makes Games, 2018.
	
	\bibitem{game:hollow}
	Team Cherry, \textit{Hollow Knight}, Team Cherry, 2017.
	
	\bibitem{game: coffee}
	Toge Productions, \textit{Coffee Talk}, Toge Productions, 2020.
	
	\bibitem{app: character}
	Character Technologies, Inc. ``character.ai'', 2022. [En línea]. Disponible: \url{https://character.ai}.  [Accedido: 11-Junio-2024].

	\end{thebibliography}
	
\end{document}
