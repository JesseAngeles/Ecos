\documentclass[12pt,twoside]{article}
\usepackage{amsmath, amssymb}
\usepackage{amsmath}
\usepackage[active]{srcltx}
\usepackage{amssymb}
\usepackage{amscd}
\usepackage{makeidx}
\usepackage{amsthm}
\usepackage{algpseudocode}
\usepackage{algorithm}
\usepackage{amssymb, amsmath}
\usepackage{fancyhdr}
\usepackage{graphics}
\usepackage{multirow}
\usepackage{graphicx}
\usepackage{graphicx}
\usepackage{caption}
\usepackage{enumitem}
\usepackage{relsize}
\usepackage{url}  % Para dar formato a las URLs
\usepackage[spanish,es-tabla]{babel}
\usepackage{listings}
\usepackage{xcolor}
\usepackage[spanish]{babel}
\usepackage{csquotes}
\usepackage{tikz}
\usetikzlibrary{snakes}
\usepackage{rotating}

% Define colores para el código
\definecolor{codegreen}{rgb}{0,0.6,0}
\definecolor{codegray}{rgb}{0.5,0.5,0.5}
\definecolor{codepurple}{rgb}{0.58,0,0.82}
\definecolor{backcolour}{rgb}{0.95,0.95,0.92}

% Configuración del estilo de Java
\lstdefinestyle{mystyle}{
	backgroundcolor=\color{backcolour},   
	commentstyle=\color{codegreen},
	keywordstyle=\color{blue},
	numberstyle=\tiny\color{codegray},
	stringstyle=\color{codepurple},
	basicstyle=\ttfamily\footnotesize,
	breakatwhitespace=false,         
	breaklines=true,                 
	captionpos=b,                    
	keepspaces=true,                 
	numbers=left,                    
	numbersep=5pt,                  
	showspaces=false,                
	showstringspaces=false,
	showtabs=false,                  
	tabsize=2
}

% Configura el estilo de Java
\lstset{style=mystyle}

%----------------------------------------------------------------------------------------------
\usepackage{amsmath, amssymb}
\usepackage{amsmath}
\usepackage[active]{srcltx}
\usepackage{amssymb}
\usepackage{amscd}
\usepackage{makeidx}
\usepackage{graphicx}
\usepackage{caption}

\renewcommand{\baselinestretch}{1}
\setcounter{page}{1}
\setlength{\textheight}{21.6cm}
\setlength{\textwidth}{14cm}
\setlength{\oddsidemargin}{1cm}
\setlength{\evensidemargin}{1cm}
\pagestyle{myheadings}
\thispagestyle{empty}
\markboth{\small{TMP(Nombre de equipo)}}{\small{Formulación del Proyecto}}
\date{}

\begin{document}
	
	\begin{center}
		
		% Contenido izquierdo - Imagen
		\begin{minipage}{0.17\textwidth}
			\flushleft
			\includegraphics[width=0.9\textwidth]{img/ipn_logo.png} % Ajusta esta línea
		\end{minipage}
		\begin{minipage}{.55\textwidth}
			\centering
			{\Large Instituto Politécnico Nacional}\\
			{\Large Escuela Superior de Cómputo}
		\end{minipage}
		\begin{minipage}{0.17\textwidth}
			\flushright
			\includegraphics[width=0.9\textwidth]{img/escom_logo} % Ajusta esta línea
		\end{minipage}			
	\end{center}
	
	
	\centerline{\bf Ingeniería en Inteligencia Artificial, Análisis y diseño de sistemas}
	
	\centerline{\bf Sem: 2024-2, 4BM2, TMP(Nombre de la actividad), Fecha: 27-06-24\\\\}
	
	\centerline{}
	
	%\centerline{}
	
	
	\begin{center}
		\Large{\textsc{Ecos del alma: Prototipo de juego rogue like con interacción jugador-NPC dinámica}} 
	\end{center}
	\centerline{}
	\centerline{\bf {\textit{Presentan}}}
	\centerline{}
	\centerline{\bf {Angeles López Erick Jesse\footnote{eangelesl1700@alumno.ipn.mx}}}
	\centerline{\bf {Corona Anchondo José Antonio\footnote{jcoronaa1900@alumno.ipn.mx}}}
	\centerline{\bf {Esquivel García Thania Paola\footnote{tesquivelg2000@alumno.ipn.mx}}}
	\centerline{\bf {	Espinosa Martínez José Carlos\footnote{jespinosam1900@alumno.ipn.mx}}}
	
	
	
	\newtheorem{Theorem}{\quad Theorem}[section]
	
	\newtheorem{Definition}[Theorem]{\quad Definition}
	
	\newtheorem{Corollary}[Theorem]{\quad Corollary}
	
	\newtheorem{Lemma}[Theorem]{\quad Lemma}
	
	\newtheorem{Example}[Theorem]{\quad Example}
	
	\bigskip
	
	\bigskip
	
	\textbf{Resumen:}  \\ 
	
	{\bf Palabras Clave:} \\
	
	\clearpage
	
	\tableofcontents
		
	\clearpage
		
	\section{Introducción}
	
	\section{Estado del arte} %Videjuegos y aplicaciones que simulen este comportamiento conversacional
	
	\begin{table}[H]
		\centering
		\begin{tabular}{|c|p{4cm}|c|}
			\hline
			Producto & Características & Precio \\ \hline
			 & &  \\ \hline
			 & &  \\ \hline
			 & &  \\ \hline
		\end{tabular}
		\caption{TMP(Resumen de productos similares)}
		\label{tabla:estado_del_arte}
	\end{table} 
	
	\section{Justificación}
	
	\section{Objetivo general}
	
	\section{Objetivos particulares}
	
	\section{TMP(Capitulos de contexto)}
	
	\section{Análisis}
	
	\subsection{Metodología}

	\subsection{Tecnología}	% Herramientas, software, lenguajes de programación que seran utilizados para el desarrollo

	\subsubsection{Lenguaje de programación C\#}
	
	\subsubsection{Lenguaje de programación Prolog}
	
	\subsubsection{Motor de desarrollo Unity}
	
	\subsubsection{MongoDB}
	
	\subsubsection{Star UML}
	
	\subsection{Requerimientos}
	
	\subsubsection{Requerimientos funcionales}
	
	\subsubsection{Requerimientos no funcionales}
	
	\subsubsection{Requerimientos de sistema}

	\section{Diseño}
	
	\subsection{Diagrama de casos de uso}
	
	\subsubsection{Especificaciones}
	
	\subsection{Diagrama de clases}

	\subsection{Diagrama de secuencia}

	\subsection{Diagrama de estados}

	\subsection{Diagrama de actividades}

	\clearpage
	
	\section{Referencias Bibliográficas}
	
	\begin{thebibliography}{10}
		

	\end{thebibliography}
	
\end{document}
